\documentclass{ctexart}
\usepackage[colorlinks=true]{hyperref}

\pagestyle{empty}

\title{Lesson 1}
\author{Nephrenn}

\begin{document}

\maketitle
\newpage

\tableofcontents
\newpage

\section{学习方法}
对于没有计划参加算法竞赛的同学, acwing 算法基础课对企业笔试面试匹配得很好,可以直接参考学习。
笔试之前到 leetcode 上刷题,基本可以满足需求。

对于有计划参加算法竞赛的同学,我觉得 acwing 算法基础课的安排是稍显不合理的。
我个人推荐的入门学习顺序是:
\begin{itemize}
    \item 贪心
    \item 二分查找、二分答案、三分
    \item dfs、bfs 和 简单剪枝
    \item 栈、队列、二叉堆、并查集
    \item 线性 dp、背包 dp、区间 dp
    \item 位运算、整除理论、组合计数
    \item 图论基础、拓扑排序、最短路、最小生成树
    \item ST 表、最近公共祖先
    \item DAG 上 dp、树形 dp、状压 dp
    \item 单调栈和单调队列
    \item 树状数组和线段树
    \item 字符串哈希、kmp 和 trie 树
\end{itemize}

关于学习方式,可以到 \href{https://www.luogu.com.cn/}{洛谷} 找以【模板】开头的题目题解学习,然后做官方和用户分享的题单中的简单题。
这部分一定不要花太多时间,争取在 12 月新生赛之前结束。

系统学习完上面的内容后,就可以开始到 \href{https://codeforces.com/}{codeforces} 上刷题了。可以在 problemset 中筛选题目难度,从 rating 1200 开始一道一道做。
遇到不会的科技就根据题解中提到的关键字去找模板题学习。如果有时间也可以打打这上面的比赛,作为自己阶段学习成果的证明。

\href{https://atcoder.jp/}{AtCoder} 上的 Atcoder Beginner Contest 是很好的锻炼自己算法模板和常见套路的地方。
这里面题通常比较简单,属于是学过了就肯定能做出来。

对高级科技有需求的,到 \href{https://oi-wiki.org/}{OIwiki} 目录里面找不会的学。比如主席树、平衡树、网络流等。在 XCPC 中一般不作为铜牌题及以下的点考察。

学习过程中可以整理一个自己的算法模板库,\href{https://github.com/nephrenn233/Templates-in-Competitive-Programming}{引流},因为 XCPC 是开卷考试。

推荐阅读:\href{https://www.zhihu.com/question/26823471/answer/2423361138}{ACM 怎么样零基础到入门? - geruome的回答 - 知乎}

\section{二分答案}
前期学习中很重要的思想。

使用条件:
\begin{itemize}
    \item 答案满足单调性。
    \item 易验证一个解是否是满足题意的。
\end{itemize}

\href{https://codeforces.com/gym/105385/problem/A}{例题: 2024 CCPC 山东邀请赛 A. Printer}

细节:二分上下界,中间返回。

\end{document}